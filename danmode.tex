\documentclass[a4paper]{article}
\usepackage{a4wide}
\pdfmapfile{=tengwarscript.map}
\usepackage[annatar]{tengwarscript}
% \usepackage[annataritalic]{tengwarscript}
% \usepackage[parmaite]{tengwarscript}
\usepackage{tipa}
%\usepackage{array}
\usepackage{multirow}
\usepackage[hidelinks]{hyperref}


\begin{document}

%\newcommand{\alen}{0.07\linewidth}
\newcommand{\tngw}[2][2]{\multirow{#1}{1em}{#2}}
\newcommand{\mystrut}{\rule[-1.0\baselineskip]{0pt}{3.5\baselineskip}}
\newcommand\T{\rule{0pt}{2.8ex}}       % Top strut
\newcommand\B{\rule[-1.4ex]{0pt}{0pt}} % Bottom strut

\begin{center}\LARGE
  \textbf{Dansk tengwar modus --- Danish tengwar mode}\\
  \tengmag{1.5}
  \Tando\Tnuumen\TTthreedots\Tsilme\Tquesse\Ts
  \Ttinco\Tungwe\TTnasalizer\TTacute\Tvala\Toore\TTthreedots\Ts
  \Tmalta\Tando\TTrightcurl\Tsilmenuquerna\TTleftcurl
\end{center}

\begin{center}
  \begin{tabular}{|ll|ll|ll|ll|}
    \hline\hline
    \multicolumn{8}{|l|}{\textbf{Konsonanter (consonants):}}\\
    \hline\hline
    \tngw{\Ttinco}&tinco&\tngw{\Tparma}&parma&
      \tngw{\Tcalma}&calma&\tngw{\Tquesse}&quess\"e \\
                  & t && p && ch, tj \textipa{[tS]}$^{\ref{noteAhaCalma}}$ && k, c [k]$^{\ref{noteCS}}$\\
    \hline
    \tngw{\Tando}&ando&\tngw{\Tumbar}&umbar&
       \tngw{\Tanga}&anga&\tngw{\Tungwe}&ungw\"e\\                                  
                  & d (hårdt [d])$^{\ref{noteD}}$ && b &&  && g (hårdt)$^{\ref{noteG}}$\\
    \hline
    \tngw{\Tthuule}&th\'ul\"e&\tngw{\Tformen}&formen&
       \tngw{\Taha}&aha&\tngw{\Thwesta}&hwesta\\                                               
    & && f, ph && sch, sh, ch \textipa{[S]}$^{\ref{noteAhaCalma}}$ && \\
    \hline
    \tngw{\Tanto}&anto&\tngw{\Tampa}&ampa&
      \tngw{\Tanca}&anca&\tngw{\Tunque}&unqu\"e\\
    & d (blødt\textipa{[D]}) $^{\ref{noteD}}$ && v && && g (blødt)$^{\ref{noteG}}$\\
    \hline
    \tngw{\Tnuumen}&n\'umen&\tngw{\Tmalta}&malta&
      \tngw{\Tnoldo}&noldo&\tngw{\Tnwalme}&nwalm\"e\\
    & n && m && && ng \textipa{[N]}$^{\ref{noteNG}}$\\
    \hline
    \tngw[3]{\Toore}&\'or\"e&\tngw[3]{\Tvala}&vala&
       \tngw[3]{\Tanna}&anna&\tngw[3]{\Tvilya}&vilya\\
                  & r $^{\ref{noteR}}$ && w$^{\ref{noteW}}$ && j$^{\ref{noteJ}}$ && \\
    & && -u diftong$^{\ref{noteDif}}$ && -i diftong$^{\ref{noteDif}}$ &&\\ 
    \hline\hline
    \tngw{\Troomen}&r\'omen&\tngw{\Tarda}&arda&
       \tngw{\Tlambe}&lamb\"e&\tngw{\Talda}&alda\\
                  & r $^{\ref{noteR}}$ && && l &&\\
    \hline
    \tngw[3]{\Tsilme}&silm\"e&\tngw[3]{\Tsilmenuquerna}&silm\"e&
       \tngw[3]{\Tesse}&ess\"e&\tngw[3]{\Tessenuquerna}&ess\"e\\
    & && ~nuquerna && && ~nuquerna\\
                  & s$^{\ref{noteCS}}$ && s, c [s]$^{\ref{noteCS}}$ && z && z \\
    \hline
    \tngw[3]{\Thyarmen}&hyarmen&\tngw[3]{\Thwestasindarinwa}&hwesta&
       \tngw[3]{\Tyanta}&yanta&\tngw[3]{\Tuure}&\'ur\"e\\
    & && ~sindarinwa && &&\\
                  & h && hv $^{\ref{noteHv}}$ &&  && \\
    \hline
    \tngw{\Tosse}&oss\"e&\tngw{\Thalla}&halla&
        \tngw{\Ttelco}&telco&\tngw{\Taara}&\'ara\\
    &  && h (stumt)$^{\ref{notehalla}}$ && vokal && (kun for å)$^{\ref{notevokal}}$  \\
    \hline\hline
    \tngw[3]{\Textendedtinco}&extended&\tngw[3]{\Textendedparma}&extended&
       \tngw[3]{\Textendedcalma}&extended&\tngw[3]{\Textendedquesse}&extended\\
                  & ~tinco && ~parma && ~calma && ~quess\"e\\
                  & th [t]$^{\ref{noteTH}}$ && ph [f] && && ch [k] $^{\ref{noteCHasK}}$\\
    \hline
    \tngw[3]{\Textendedando}&extended&\tngw[3]{\Textendedumbar}&extended&
       \tngw[3]{\Textendedanga}&extended&\tngw[3]{\Textendedungwe}&extended\\
                  & ~ando && ~umbar && ~anga && ~ungw\"e\\
                  & d (stumt)$^{\ref{noteD}}$ &&  && && \\
    \hline\hline
    \multicolumn{8}{|l|}{\textbf{Vokaler (vowels):}}\\
    \hline\hline
    \Ttelco\TTthreedots& a & \Ttelco\TTacute & e & 
       \Ttelco\TTdot & i & \Ttelco\TTrightcurl & o \T\B \\
    \hline
    \Ttelco\TTleftcurl & u & \Ttelco\TTbreve & y &
       \Ttelco\TTinvertedthreedots & æ$^{\ref{notevokal}}$ \T\B& \Ttelco\TTdoubleacute & ø$^{\ref{notevokal}}$ \\
    \hline
    \Taara\TTthreedots& å$^{\ref{notevokal}}$ & \Ttelco\TTtwodots & \"u$^{\ref{notevokal}}$ &
       \Ttelco\TTdotbelow & e (stumt) \T\B& & \\
    \hline\hline
  \end{tabular}
\end{center}

% UNCOMMENT TO TEST Æ-TETHA PLACEMENT
% \begin{center}
%  \tengmag{3}
%  \Taara\TTthreedots\Tvala\TTthreedots\Tvala\TTinvertedthreedots\Tvala\TTrightcurl\Tvala\TTdot\Tvala\TTacute\Tvala\TTthreedots
% \end{center}

\pagebreak
{\Large \textbf{Notes (English):}\par}

\begin{enumerate}
\item Danish vowels Æ, Ø, and Å: \Ttelco\TTinvertedthreedots\Ts
  \Ttelco\TTdoubleacute ~and \Taara\TTthreedots, respectively.  The
  first two follow the example of the German and Swedish modes.  The Å
  reflects the pre-1948 orthography of writing the letter as Aa.  Note
  that even then, aa was considered a distinct vowel alphabetized at
  the end of the alphabet.  The long carrier (\'ara) should not be
  used for other purposes.  Y is always a vowel.  The german letter
  \"u sometimes appears in names, and can be written as in the German
  mode \Ttelco\TTtwodots ~or as y \Ttelco\TTbreve ~according to
  taste.\label{notevokal}
\item Vowel combinations like ue can be two distinct vowels written as
  such, or the e can be silent and be written with the e-dot below: \emph{due}
  \Tando\Ttelco\TTleftcurl\Ttelco\TTacute ~(pigeon) but
  \emph{vindue}
  \Tampa\Tando\TTnasalizer\TTdot\Ttelco\TTleftcurl\TTdotbelow
  ~(window). 
\item Two juxtaposed vowels are usually just that in Danish, but in
  the few cases where it is a diphthong they can be 
  written as in the English mode: Diphthongs ending in -u and -i are written
  with a tetha above vala and anna, respectively.  There are probably
  no diphthongs ending in -e and -a, but if there are, they can be
  written with yanta and oss\"e.\label{noteDif}
\item D is pronounced in three equally common ways: the ``hard d''
  ([d], clearly ando \Tando), the ``soft d'' (\textipa{[D]}, clearly anto \Tanto), and the
  ``silent d'' where extended ando
  \Textendedando ~is proposed.\label{noteD}
\item G can also be ``hard'' or ``soft''.  Whereas the ``hard g'' is
  close to the same sound in English (ungw\"e \Tungwe), the so-called
  soft g is an ill-defined continuum of weak sounds reaching from
  completely silent, over semivowels \textipa{[I]} and \textipa{[U]},
  to an sh \textipa{[S]} in loan words from English or French (where
  we attempt to preserve to pronunciation, but the original voiced sound
  \textipa{[Z]} does not exist in Danish).  While the soft G
  probably deserves several different tengwar, it is suggested to lump
  it all into unqu\"e \Tunque.\label{noteG}
  \item NG is usually the sound \textipa{[N]} and written with
    nwalm\"e.  Occationally, it is \textipa{[Ng]} and is written with
    nasalized ungwe: \emph{finger}
    \Tformen\Tnwalme\TTdot\Toore\TTacute ~(a finger) but
    \emph{fingere}
    \Tformen\Tungwe\TTnasalizer\TTdot\Troomen\TTacute\Ttelco\TTacute
    ~(to feign).\label{noteNG}
\item Th only appear in names, where it is pronounced as T and written
  with extended tinco \Textendedtinco.  The th\'ul\"e sound
  \textipa{[T]} does not exist in Danish.\label{noteTH}
\item J is pronounced as consonant Y in English, and written with anna
  \Tanna.  The letter combination ai and ei has in modern orthography
  been replaced with aj and ej, but still appears in names with the
  same pronunciation.  They can therefore be written as two vowels or
  with anna, according to taste: The name \emph{Mai} is
  \Tmalta\Ttelco\TTthreedots\Ttelco\TTdot ~or \Tmalta\Tanna\TTdot
  ~whereas the month \emph{maj} is written
  \Tmalta\Tanna\TTdot.\label{noteJ}
\item The sound \textipa{[S]} appears in German-derived names starting
  with Sch, and in a few borrowed words with sh- or ch- like
  \emph{chokolade}.  It can be written with aha:
  \Taha\Tquesse\TTrightcurl\Tlambe\TTrightcurl\Tanto\TTthreedots\TTdotbelow.
  Ch can also be the same sound as tj
  \textipa{[tS]} (often even with a choice in orthography), both are
  written with calma \Tcalma.\label{noteAhaCalma}  \emph{Tjekkiet}
  \Tcalma\Tquesse\TTacute\TTdoubler\Ttelco\TTdot\Ttinco\TTacute\Ts
  (Czechia).
\item CH pronounced as K is written with extended quesse.
  \emph{Michael}
  \Tmalta\Textendedquesse\TTdot\Ttelco\TTthreedots\TTdotbelow\Tlambe.
  See also note \ref{noteAhaCalma}. \label{noteCHasK}
\item S and C: S is written with silm\"e, or silm\"e nuquerna if that
  makes theta placement more convenient, but should never be written
  with silm\"e nuquerna if there is no tetha. At the end of the word
  an s-hook (sa-rinc\"e) can be used, if desired. S is never voiced,
  and therefore never written with ess\"e.  C is quite rare; it
  can be pronounced [k] or [s] and is written with quesse or silm\"e
  nuquerna, respectively.  Silm\"e nuquerna without a tetha will thus
  be C, with a tetha it is with high probability an s.  \emph{Cirkus}:
  \Tsilmenuquerna\Toore\TTdot\Tquesse\Tsilmenuquerna\TTrightcurl.\label{noteCS}
  The letter Z is pronounced exactly like S in Danish, but should still
  be written with ess\"e or ess\"e nuquerna.
\item Hv: Words starting with hv are written with hwesta sindarinwa,
  although the h is completely silent.  This is to acknowledge the
  common origin and almost one-to-one mapping with the wh- words in
  English (e.g. hvad \Thwestasindarinwa\Tanto\TTthreedots ~= what,
  hvor  = where, hvid = white).\label{noteHv}
\item Hj: Words starting with hj are written with halla anna as the H
  is completely silent.  No single tengwa is used here, as there is no
  mapping to similar words in other languages where a single tengwa
  would be used.  In the contrary, these words map almost one-to-one
  to Icelandic (close to Old Norse), where hj is pronounced as two
  consecutive consonant sounds.  Presumably, Icelandic elves would
  write these words with hyarmen anna, which is shifted to halla anna
  in Danish to reflect that the H is silent.  Other words with silent
  h are also written with halla instead of hyarmen.\label{notehalla}
\item W is not officially in the Danish alphabet, but is common in
  names where it is pronounced as V.  Nevertheless, it makes sense to
  distinguish.  Vala can also be used for the rare u-glide diphthongs:
  \emph{Laura:} \Tlambe\Tvala\TTthreedots\Troomen\Ttelco\TTthreedots.\label{noteW}
\item The Danish mode follows the r-rule: \'or\"e is used before a
  consonant and at the end of the word, r\'omen is used before a
  vowel.  A following word starting with a vowel does \emph{not}
  change a final \'or\"e to r\'omen, as it does not change the
  rhoticity of the r (Danish r's are in general not very rhotic).\label{noteR}
\item Compound words: Terms consisting of multiple nouns in English
  are compound words in Danish (office chair, mobile phone, \ldots).
  Although it is a travesty to write such words with a hypen in
  Danish, the compoundness should be respected when writing with
  tengwar: A final vowel (or nasal sound) from one part does not
  ``climb onto'' the first consonant of the following part.
  \emph{Kanotur} (canoe trip) is
  \Tquesse\Tnuumen\TTthreedots\Ttelco\TTrightcurl
  \Ttinco\Toore\TTleftcurl ~not
  \Tquesse\Tnuumen\TTthreedots\Ttinco\TTrightcurl\Toore\TTleftcurl.
\item English loan words and names are written according to the
  English mode, if the original pronunciation is approximately
  retained.  \emph{Jane} is written
  \Tanga\Tnuumen\TTthreedots\TTdotbelow ~or
  \Tanna\Tnuumen\TTthreedots\Ttelco\TTacute, depending on how she
  pronounces it.
\end{enumerate}

{\Large \textbf{Noter (Dansk):}\par}

\begin{enumerate}
\item Æ, Ø og Å skrives som \Ttelco\TTinvertedthreedots\Ts
  \Ttelco\TTdoubleacute ~og \Taara\TTthreedots.  De første to følger
  \"a og \"o i svensk og tysk modus.  Å er inspireret af aa fra før
  retskrivningsreformen i 1948.  Den lange vokalbærer (\'ara) bør ikke
  bruges til andre formål.  Tysk \"u (fx i navne) kan skrives om i den
  tyske modus  \Ttelco\TTtwodots ~eller som y \Ttelco\TTbreve, efter
  behag.
\item Vokalkombinationer som ue kan være to vokaler, eller e'et kan
  være stumt og skrives med e-prik under den anden vokal: \emph{due}
  \Tando\Ttelco\TTleftcurl\Ttelco\TTacute ~men \emph{vindue}
  \Tampa\Tando\TTnasalizer\TTdot\Ttelco\TTleftcurl\TTdotbelow.
\item To vokaler er oftest blot to vokaler, man kan være en diftong, i
  så fald skrives de som i den engelske modus: diftonger som ender på
  -u og -i skrives med vala og anna: \emph{Laura}
  \Tlambe\Tvala\TTthreedots\Troomen\Ttelco\TTthreedots, \emph{Leif}
  \Tlambe\Tanna\TTacute\Tformen.
\item Hårdt D er ando \Tando, blødt D er anto \Tanto, og stumt D er
  extended ando \Textendedando.
\item Hårdt G er ungw\"e \Tungwe.  Blødt G kan være mange lyde på
  dansk, fra helt stumt over halvvokalerne \textipa{[I]} og
  \textipa{[U]}, til sh lyd \textipa{[S]} i låneord.  Det foreslås at
  det hele slås sammen i unqu\"e \Tunque.
\item NG er normalt lyden \textipa{[N]} og skrives med nwalm\"e
  \Tnwalme.  I enkelte ord udtales ng med separat g-lyd \textipa{[Ng]}, og skrives med
  nasaliseret ungwe \Tungwe\TTnasalizer.  En \emph{finger}
  \Tformen\Tnwalme\TTdot\Toore\TTacute, men at \emph{fingere}
  \Tformen\Tungwe\TTnasalizer\TTdot\Troomen\TTacute\Ttelco\TTacute
  ~(som i \emph{et fingeret mord}).
\item Th optræder vist kun i navne som Thomas med lyden T, og skrives
  med extended tinco \Textendedtinco ~som i engelsk modus.  Lyden
  svarende til th\'ul\"e \textipa{[T]} findes ikke på dansk.
\item J er anna \Tanna, dog kan anga \Tanga ~bruges i engelske navne
  hvis de udtales på engelsk.  Bogstavkombinationerne ai og ei bruges
  stadigt i personnavne, men er ellers erstattet med aj og ej.  I
  navne kan de skrives som to tegn eller en diftong: navnet \emph{Mai}
  er enten \Tmalta\Ttelco\TTthreedots\Ttelco\TTdot ~eller
  \Tmalta\Tanna\TTdot ~(efter smag), måneden \emph{maj} er
  \Tmalta\Tanna\TTdot.\label{noteJ}
\item Lyden \textipa{[S]} optræder fx i tyske og tyskinspirerede navne
  som Sch-, og i enkelte ord startende med sh- og ch-.  Lyden skrives
  med aha: \emph{chokolade}
  \Taha\Tquesse\TTrightcurl\Tlambe\TTrightcurl\Tanto\TTthreedots\TTdotbelow.
  Ch kan også udtales tj \textipa{[tS]} (ofte med valgfri stavemåde),
  den lyd skrives med calma \Tcalma.  \emph{Tjekkiet}
  \Tcalma\Tquesse\TTacute\TTdoubler\Ttelco\TTdot\Ttinco\TTacute.
\item CH udtalt som K skrives med extended quesse.  \emph{Michael}
  \Tmalta\Textendedquesse\TTdot\Ttelco\TTthreedots\TTdotbelow\Tlambe.
  Se også note \ref{noteAhaCalma}.
\item S og C: S skrives med silm\"e \Tsilme, eller silm\"e nuquerna
  for at give plads til en tetha, men bør aldrig skrives med silm\"e
  nuquerna hvis der ikke er en tetha.  Sidst i et ord kan en s-krog
  (sa-rinc\"e, \Tempty\Trighthook) bruges.  C udtalt [k] skrives med quess\"e
  \Tquesse, C udtalt [s] med silm\"e nuquerna \Tsilmenuquerna.
  Silm\"e nuquerna uden en tetha er derfor altid C, med en tetha
  oftest S.  \emph{Cirkus}:
  \Tsilmenuquerna\Toore\TTdot\Tquesse\Tsilmenuquerna\TTrightcurl. Z
  skrives med ess\"e \Tesse ~eller ess\"e nuquerna \Tessenuquerna.
\item Hv først i et ord skrives med hwesta sindarinwa
  \Thwestasindarinwa ~for at afspejle at der er en fælles oprindelse
  og en en-til-en match mellem danske hv-ord og engelske wh-ord: hvad
  \Thwestasindarinwa\Tanto\TTthreedots ~= what, hvor = where, hvid =
  white.
\item Hj først i et ord skrives med halla anna \Thalla\Tanna ~da h-et
  er stumt, og da der ikke er en tilsvarende match med et sprog, hvor
  hj naturligt skrives som \'et tegn.  Tværtimod er der en fin match med
  islandsk (tæt på oldnordisk), hvor de samme ord findes, og hvor både
  h og j udtales (tungebrækker!).  Men da h er stumt på dansk skrives halla
  anna, ikke hyarmen anna.  Halla kan også bruges til andre stumme
  h-er.
\item W (fx i navne) skrives med vala, selvom det udtales som V.  Vala
  kan også bruges til -u diftonger: \emph{Laura:}
  \Tlambe\Tvala\TTthreedots\Troomen\Ttelco\TTthreedots.
\item Dansk modus følger r-reglen: \'or\"e \Toore ~bruges før en
  konsonant og sidst i et ord (også før stumt e), r\'omen \Troomen
  ~bruges før en vokal.
\item  Sammensatte ord respekterer de enkelte dele: en vokal fra sidst
  i et led klatrer ikke op på første konsonant i næste led. \emph{Kanotur} er
  \Tquesse\Tnuumen\TTthreedots\Ttelco\TTrightcurl
  \Ttinco\Toore\TTleftcurl ~ikke
  \Tquesse\Tnuumen\TTthreedots\Ttinco\TTrightcurl\Toore\TTleftcurl.
\item Engelske låneord og navne staves efter den engelske modus, hvis
  den engelske udtale er bevaret.  \emph{Jane} skrives
  \Tanga\Tnuumen\TTthreedots\TTdotbelow ~eller
  \Tanna\Tnuumen\TTthreedots\Ttelco\TTacute, afhængig af hvordan hun
  udtaler det.
\end{enumerate}

{\Large \textbf{References:}\par}
\begin{itemize}
\item Per Lindberg: ``At skriva svenska med tengwar'',
  \url{https://www.forodrim.org/daeron/teng-swe.pdf}
\item Alexander K\"orschgen: ``Deutscher Tengwar-Modus'',\\
  \url{https://web.archive.org/web/20140227102534/http://my.opera.com/tengwarblog/blog/tengwar-orthografischer-deutscher-tehtar-modus}
\end{itemize}
\end{document}

%%% Local Variables:
%%% mode: latex
%%% TeX-master: t
%%% End:
